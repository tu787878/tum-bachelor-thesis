\chapter{Inductive statements for regular transition system}\label{chapter:inductive_statement}
\paragraph*{}
RMC considers the systems, in which the system's states can be modelled as regular words 
over a alphabet.

In \autoref{section:rts}, we introduce \textit{Regular transition system} (RTS)
as a model of RMC to represent the behavior of a parameterized system.
\autoref{section:inductive_statements} will examine the process of how the statements are encoded.

\section{Regular transition system}\label{section:rts}
The RMC program model consists of a finite alphabet $\Sigma$, 
a set of initial configurations represented by a regular set over $\Sigma$, 
and regular transition relations between words on $\Sigma^*$.
For instance, in any parameterized systems $\mathcal{S}$ for linear topology with $n$ processes, 
the set of finite states for each process is defined as the alphabet $\Sigma$.
The configurations of this system are presented by the set of words over the alphabet $\Sigma$.
By setting the length of these configurations to the number of processes, 
each word represents the corresponding process state.
For example, a configuration "$u_1 \, u_2 \, \dots \, u_n$" with $u_1,\dots,u_n \in \Sigma$ indicates that the first process 
has state $u_1$, the second process has state $u_2$, and so on.

Another important point to note is that, in RMC, the system $\mathcal{S}$ defines some semantics, 
including the regular relation between configurations. 
In other words, this relation captures the behavior of the system by defining 
how processes on the system can change states.
Formally, we can use a finite-state transducer over $\Sigma \times \Sigma$ to recognize this relation:

\newpage
\begin{theo}[Transducer]{definition:transducer}
    \textit{
        A $\Sigma$-$\Gamma$-\textit{transducer} $\mathcal{T}$ is an \textit{NFA} 
        $\langle Q, Q_0, \Sigma \times \Gamma, \Delta, F \rangle$,
        we denote for any relations 
        \begin{equation*}
          \langle u_1 \dots u_n, v_1 \dots v_n \rangle \in \bigcup_{n \geq 0} \Sigma^{n} \times \Gamma^{n} 
        \end{equation*}
        if and only if 
        \begin{equation*}
            \langle u_1, v_1 \rangle, \dots, \langle u_n, v_n \rangle \in \mathcal{L}(\mathcal{T})
        \end{equation*}
        Note that the transducer is length-preserving which means that has the capability to only associate words that have equal length.
        With $[[\mathcal{T}]]$ as a set of these relations, we define
        \begin{align*} 
            For \,\, v \in \Sigma^*: target_{\mathcal{T}}(v) = \lbrace u \in \Gamma^*\,\, | \,\, \langle v,u \rangle \in [[\mathcal{T}]] \rbrace 
        \end{align*}
    }
\end{theo}

As previously mentioned, we will utilize the RTS as a model for the RMC throughout this thesis.
\begin{theo}[Regular transition system (RTS)]{definition:rts}
    \textit{
    An RTS is a triple $\mathcal{R} = \langle \Sigma, \mathcal{I}, \mathcal{T} \rangle$ where
    }
    \begin{itemize}
        \item \textit{$\Sigma$ is a finite alphabet including all states,}
        \item \textit{$\mathcal{I}$ is an NFA, which recognizes the language of initial configurations,}
        \item \textit{$\mathcal{T}$ is a $\Sigma$-$\Sigma$-\textit{transducer} of the system, which recognizes the transition relation of the system.}
      \end{itemize}
\end{theo}

For every pair of configurations $\langle u, v \rangle$, $u \rightsquigarrow_\mathcal{T} v$ implies $\langle u, v \rangle \in [[\mathcal{T}]]$.
Moreover, the reflexive transitive closure of $\rightsquigarrow_\mathcal{T}$ is denoted with $\rightsquigarrow_\mathcal{T}^{*}$.
From any $u \in \mathcal{L}(\mathcal{I})$, we call $w$ is $reachable$ in $\mathcal{R}$ if one
exists $u \rightsquigarrow_\mathcal{T}^{*} v$. We denote $reach(\mathcal{R}) \in \Sigma^*$ for all such $reachable$ configurations.

\begin{example}
    \textit{Token passing}
\end{example}
We will provide a simple example to demonstrate how parameterized systems are 
modelled in \textit{RTS}.
The token-passing system is a sequence of agents arranged in a straight line. 
Each agent is capable of possessing only one token and can transfer it to the adjacent agent on the right. 
We assume that there is only one agent that can have the token at any given time. 
At the start, the first agent holds the token. 
With each turn, the agent who currently holds the token can pass it to the next agent on the right.

To represent the system as an RTS, we denote the agent that holds the token with "t" and the one that doesn't with "n". 
Formally, we use the alphabet $\Sigma = \{n, t\}$ for our system.
The language $\mathcal{L}(\mathcal{I})$ in Automaton \ref{lem:initial} defines the set of initial configurations for the \textit{token passing} system
as tn*.
In other words, the first agent always holds the token, while the following agents do not.

\begin{lem}[NFA $\mathcal{I}$ for Token passing]{lem:initial}
    \textit{Automaton $\mathcal{I}$ over the alphabet $\Sigma = \{n,t\}$}.\\
    \begin{tikzpicture}[shorten >=1pt,node distance=2cm,on grid,auto] 
        \node[state,initial] (q_0)   {$q_0$}; 
        \node[state,accepting] (q_1) [right=of q_0] {$q_1$}; 
         \path[->] 
         (q_0) edge  node {t} (q_1)
         (q_1) edge [loop above] node {n} ();
     \end{tikzpicture}
\end{lem}

Transducer $\Gamma$ in Automaton \ref{lem:transducer} recognizes the transition relation of the token passing system.
Intuitively, the token will be transferred from the left agent to the right agent. 
Once the token reaches the end of the agents, no further transitions can be made to the configuration.
\begin{lem}[Transducer $\Gamma$ for Token passing]{lem:transducer}
    \textit{Automaton $\mathcal{T}$ over the alphabet $\Sigma \times \Sigma = \left\{[ n,n ], [ n,t ], [ t,n ], [ t,t ]\right\}$}. \\
    \begin{tikzpicture}[shorten >=1pt,node distance=2cm,on grid,auto] 
        \node[state,initial] (p_0)   {$p_0$}; 
        \node[state] (p_1) [right=of p_0] {$p_1$}; 
        \node[state,accepting] (p_2) [right=of p_1] {$p_2$}; 
         \path[->] 
         (p_0) edge  node {$[t, n]$} (p_1)
               edge [loop above] node {$[n, n]$} ()
         (p_1) edge  node {$[n, t]$} (p_2)
         (p_2) edge [loop above] node {$[n, n]$} ();
     \end{tikzpicture}
\end{lem}

\paragraph*{}
We capture the transitions of the \textit{token passing} system via the language 
$[n,n]^*[t,n][n,t][n,n]^*$.
Consider the following scenario: "n n t n" represents a system with four agents, 
where the third agent currently holds the token. In the next step, 
only the third agent can transfer the token to the agent on the right. 
This can be represented as "n n n t". The system terminates since the last agent 
holds the token.

We assume that the DFA $\mathcal{B}$ (\ref{lem:manytoken}) represents a set of "unsafe" configurations of the token passing system.
In other words, all the configurations that have more than one agent hold the token must not be reached in $\mathcal{S}$. 
For instance, we must guarantee that $u \not\rightsquigarrow_\mathcal{T}^{*} v$ for all $u \in \mathcal{L}(\mathcal{I})$ and $v \in \mathcal{L}(\mathcal{B})$.
\newpage
\begin{lem}[DFA $\mathcal{B}$ for "manytoken"]{lem:manytoken}
    \textit{Automaton $\mathcal{B}$ over the alphabet $\Sigma = \{n,t\}$}.\\
    \begin{tikzpicture}[shorten >=1pt,node distance=2cm,on grid,auto] 
        \node[state,initial] (q_0)   {$q_0$}; 
        \node[state] (q_1) [right=of q_0] {$q_1$}; 
        \node[state, accepting] (q_2) [right=of q_1] {$q_2$}; 
         \path[->] 
         (q_0) edge  node {t} (q_1)
               edge [loop above] node {n} ()
         (q_1) edge [loop above] node {n} ()
         (q_1) edge  node {t} (q_2)
         (q_2) edge [loop above] node {n,t} ();
     \end{tikzpicture}
\end{lem}

\section{Inductive statements}\label{section:inductive_statements}
In general, we need to determine if a given RTS can produce any undesirable word,
which is pre-defined in a regular set.
We call this the \textit{Reachability problem}.
\subsection*{Reachability problem}
Assume that $\mathcal{L}(\mathcal{B})$ is a set of undesired configurations for a \textit{RTS} $\mathcal{R}$.
The question at hand is whether it is possible to reach any undesired configurations from the initial configurations.
Formally, we have to compute
if \textit{reach}($\mathcal{R}$) $\bigcap$ $\mathcal{L}(\mathcal{B}) = \emptyset?$
Because the reachability problem is undecidable \cite{bloem2016decidability}, we propose a semi-decision procedure that can provide an answer to the following question:
"whether there exists a pair of configurations, $v$ and $u$, 
where $u$ satisfies all the inductive statements that $v$ satisfies"?

Indeed, if $u$ is reachable from $v$, it will satisfy all the inductive statements that $v$ does since they are inductive. 
By using inductive statements one can ensures that no undesired configurations can be reached from any initial configuration. 
That means that since a reachable configuration is not part of the undesired language, it satisfies all the inductive statements that the initial does. 
In simpler terms, at least one inductive statement is satisfied by the initial configuration but not the undesired one.

\subsection*{Encoded statements}
We shall now proceed to examine the process of how the statements are encoded.
Recall that, the RMC represents the configurations of the system with finite words. 
In this representation, the i-th word of the configuration indicates the state of the i-th process in the system. 
The only necessary information required is the index and the state symbol.
Formally, we encode the statement information as the function $f$: $\{1,\dots,m\} \rightarrow 2^\Sigma$, 
where $m$ is the number of processes and $f(i) \in \Sigma$ corresponds to the set of possible states of i-th process.
For example, we assume a system with the alphabet $\Sigma = \{p,q\}$ and a statement in which all configurations must have a length of 4.
Moreover, the first process must have state p or q while the second must have state p.
Then, the statement information can be translated to $\{1 \rightarrow \{p, q\}, \, 2 \rightarrow \{p\}, \, 3 \rightarrow \emptyset,\, 4 \rightarrow \emptyset\}$.
The function $f$: $\{1,\dots,m\} \rightarrow 2^\Sigma$ is injective, which means
that we can uniquely determine a set of states from $2^\Sigma$ for every index $\{1,\dots,m\}$.
In other words, we can represent the statement as a word with a length of m over the alphabet $\Sigma$.
For instance, a statement $\{p, q\} \,\, \{p\} \,\, \emptyset \,\,\emptyset$ represents a function $\{1 \rightarrow \{p, q\}, \, 2 \rightarrow \{p\}, \, 3 \rightarrow \emptyset,\, 4 \rightarrow \emptyset\}$.

Without context, these words lack any meaning or explanation.
Therefore, we explore the \textit{interpretations} to decide whether the configuration satisfies the statement.

\begin{theo}[Interpretation]{definition:interpretation}
    \textit{
        For any RTS $\mathcal{R} = \langle \Sigma, \mathcal{I}, \mathcal{T} \rangle$, we call 
        a pair $\langle \Gamma, \mathcal{V} \rangle$ an $\Gamma$-\textit{interpretation}
        where $\Gamma$ is a finite alphabet and $\mathcal{V}$ is a deterministic $\Sigma$-$\Gamma$-\textit{transducer}.
        To express that the pair $\langle u, I \rangle$ belongs to the interpretation of $\mathcal{V}$, we use the notation $u \models I$ and it means that u satisfies I.
        Formally,
     }
     \begin{equation*}
        \langle u, I \rangle \in [[\mathcal{V}]] \Leftrightarrow u \models I
     \end{equation*}
\end{theo}
Assume that $u = u_1 \dots u_n \in \Sigma$ and $I = I_1 \dots I_n\in 2^\Sigma$.
The interpretations rely on the satisfaction of atomic propositions $u_i \in I_i$ with $1 \leq i \leq n$ 
to determine if the configuration $u$ satisfies the statement $I$.
For the sake of clarity, we have outlined three different interpretations: $trap$, $siphon$, $flow$.

\paragraph*{Traps} 
In $trap$ interpretation, the method to determine if a configuration $u$ satisfies a statement $I$ 
involves checking if at least one atomic proposition is true in the proposition. 
If it is, then it can be concluded that $u$ satisfies the statement $I$. 
In other words, if one exits $u_i \in I_i$ for any $i$ with $1 \leq i \leq n$ then $u \models_{\mathcal{V}_{trap}} I$.
To illustrate, let's examine whether the configuration \textit{"n n n n"} satisfies the statement \{t\} \{n,t\} $\emptyset$ $\emptyset$ by the $trap$ interpretation?
We can check this by evaluating the proposition: $n \in \{t\} \vee n \in \{n,t\} \vee n \in \emptyset \vee n \in \emptyset$.
Intuitively, we can conclude that $u$ satisfies $I$ or $u \models_{\mathcal{V}_{trap}} I$ since this proposition is true.

We construct the Automaton \ref{lem:trap} as a $\Sigma$-$\Gamma$-\textit{transducer} to recognize all pairs of the configuration $u$ and the statement $I$
if $u \models_{\mathcal{V}_{trap}} I$.
\newpage
\begin{lem}[DFA for trap intepretation]{lem:trap}
    \textit{We denote with H all pairs in $\langle \sigma, I \rangle \in \Sigma \times \Gamma$
    such that  $\sigma \in I$ and M all pairs in $\langle \sigma, I \rangle \in \Sigma \times \Gamma$
    such that  $\sigma \notin I$.}

    \begin{tikzpicture}[shorten >=1pt,node distance=2cm,on grid,auto] 
        \node[state,initial] (q_0)   {$q_0$}; 
        \node[state, accepting] (q_1) [right=of q_0] {$q_1$}; 
         \path[->] 
         (q_0) edge  node {H} (q_1)
               edge [loop above] node {M} ()
         (q_1) edge [loop above] node {H, M} ();
     \end{tikzpicture}
\end{lem}

\paragraph*{Siphon}
Under the siphon interpretation, none of the atomic propositions hold true.
To formalize this, we can take into account the proposition: $u_1 \notin I_1 \wedge \dots \wedge u_n \notin I_n$.
When this proposition holds true, we can infer that $u$ satisfies the statement $I$.

\begin{lem}[DFA for siphon intepretation]{lem:siphon}
    \textit{Again, we denote with H all pairs in $\langle \sigma, I \rangle \in \Sigma \times 2^{\Sigma}$
    such that  $\sigma \in I$ and M all pairs in $\langle \sigma, I \rangle \in \Sigma \times 2^{\Sigma}$
    such that  $\sigma \notin I$.}

    \begin{tikzpicture}[shorten >=1pt,node distance=2cm,on grid,auto] 
        \node[state,initial, accepting] (q_0)   {$q_0$}; 
        \node[state] (q_1) [right=of q_0] {$q_1$}; 
         \path[->] 
         (q_0) edge  node {H} (q_1)
               edge [loop above] node {M} ()
         (q_1) edge [loop above] node {H, M} ();
     \end{tikzpicture}
\end{lem}

\paragraph*{Flow}
This time we want only one atomic proposition is satisfied.
For instance, given the configuration $u = $ "n n n t", $v = $ "t t t t" and the statement $I =$ $\{t\} \,\, \{t\} \,\, \{t\} \,\, \{n, t\}$,
one can conclude that $u \models_{\mathcal{V}_{flow}} I$ and $v \not\models_{\mathcal{V}_{flow}} I$.
\newpage
\begin{lem}[DFA for flow intepretation]{lem:flow}
    \textit{We also denote with H all pairs in $\langle \sigma, I \rangle \in \Sigma \times 2^{\Sigma}$
    such that  $\sigma \in I$ and M all pairs in $\langle \sigma, I \rangle \in \Sigma \times 2^{\Sigma}$
    such that  $\sigma \notin I$.}

    \begin{tikzpicture}[shorten >=1pt,node distance=2cm,on grid,auto] 
        \node[state,initial] (q_0)   {$q_0$}; 
        \node[state, accepting] (q_1) [right=of q_0] {$q_1$}; 
        \node[state] (q_2) [right=of q_1] {$q_2$}; 
         \path[->] 
         (q_0) edge  node {H} (q_1)
               edge [loop above] node {M} ()
         (q_1) edge [loop above] node {M} ()
               edge  node {H} (q_2)
         (q_2) edge [loop above] node {H,M} ();
     \end{tikzpicture}
\end{lem}


\begin{theo}[Inductive statements]{definition:inductive_statements}
    \textit{
        Assume that $\mathcal{V}$ is the $\Gamma$-\textit{interpretation} for $\mathcal{R} = \langle \Sigma, \mathcal{I}, \mathcal{T} \rangle$, we define
    }
    \begin{gather*}
        Inductive_{\mathcal{V}}(\mathcal{R}) = \left\lbrace I \in \Gamma^* \, | \,
        \forall u \rightsquigarrow_\mathcal{T} v. \,\, if \,\, \langle u, I \rangle
        \in [[\mathcal{V}]] \,\, then \,\, \langle v, I \rangle \in [[\mathcal{V}]] \right\rbrace \\
        =  \lbrace I \in \Gamma^* \, |  \,
        \forall u \rightsquigarrow_\mathcal{T} v. \,\, if \,\, u \models I
            \,\, then \,\, v \models_{\mathcal{V}} I \rbrace
    \end{gather*}
\end{theo}
Inductively, the statement $I$ will remain satisfied throughout every possible transition of the RTS.
Thus, every configuration reachable from any $v$ also satisfies all the statements that $v$ satisfies.
Formally, when $v \rightsquigarrow_\mathcal{T}^* w$ if $v \models_{\mathcal{V}} I$ then also $w \models_{\mathcal{V}} I$.
In other words, we can provide $v \not\rightsquigarrow_\mathcal{T}^* u$ if there exists a statement $I$ such that
$v \models_{\mathcal{V}} I$ but $u \not\models_{\mathcal{V}} I$.

\begin{example}
    Inductive statements of token-passing system, $\mathcal{V}_{trap}$.
\end{example}
To illustrate, let us examine the example:
\[
    \{n\} \,\, \{n\} \,\, \{n\}^* \,\, (\{t\} \,\, \{n\} \,\, \{n\}^* \,\, | \,\, \{t\}) \footnote{The regular set has been learned in practice by dodo-cpp using token-passing system with trap interpretation, and the "manytoken" property.}
     \subseteq Inductive_{\mathcal{V}_{trap}}(\mathcal{R}) 
\]
For the fixed length of configuration n = 4, all possible inductive statements are
\begin{gather*}
    \lbrace n \rbrace\,\,\,\, \lbrace n \rbrace \,\,\,\,  \lbrace n \rbrace \,\,\,\, \lbrace n \rbrace, \label{exm:ind_statements1}\\ 
    \lbrace n \rbrace\,\,\,\, \lbrace n \rbrace \,\,\,\,  \lbrace n \rbrace \,\,\,\, \lbrace t \rbrace, \label{exm:ind_statements2}\\
    \lbrace n \rbrace\,\,\,\, \lbrace n \rbrace \,\,\,\,  \lbrace t \rbrace \,\,\,\, \lbrace n \rbrace \label{exm:ind_statements3}
\end{gather*}
We can observe that every possible configuration that can be reached from the initial configuration $u = $ "\textit{t n n n}" also satisfies all the statements that the initial configuration satisfies.
Intuitively, a configuration $w = $ "\textit{n n t n}" can be reached from $u$ both satisfy all above statements, which are also satisfied by $u$.
It also concludes that $v = $ "\textit{t t t n}" $\in \mathcal{L}(\mathcal{B})$ can not reached from the initial configuration
$u = $ "\textit{t n n n}" $\in \mathcal{L}(\mathcal{I})$ because $u$ satisfies the statement
$I = \lbrace n \rbrace\,\, \lbrace n \rbrace \,\,  \lbrace n \rbrace \,\, \lbrace t \rbrace$,
but $v$ does not.

By this way, we can guarantee that no bad configurations can be reached
by checking the satisfaction of the configurations with the statements.

\begin{theo}[Potential reachability]{definition:potential_reachability}
    \textit{
   Let $\mathcal{R} = \langle \Sigma, \mathcal{I}, \mathcal{T} \rangle$ be any \textit{RTS}
   and $\langle \Gamma, \mathcal{V} \rangle$ any interpretation.
   We write $v \Rightarrow_{\mathcal{V}} I$ if and only if $u \models_{\mathcal{V}} v$
   for all $I \in target_\mathcal{V}(u) \bigcap Inductive_\mathcal{V}(\mathcal{R})$.
   }
\end{theo}

\begin{lemma}\label{lemma:abstractly_reachable}
    
\end{lemma}
\textit{
    Let $\mathcal{R} = (\Sigma, I, T)$ be an RTS, $\langle \Gamma, \mathcal{V} \rangle$ an interpretation, 
    and S a NFA over the alphabet $\Gamma$. Then there exists a  $\Sigma-\Sigma-transducer$ C such that
}
\begin{equation}
    [[C]] = \left\lbrace \langle u,v\rangle \in \bigcup_{n \geq 0} \Sigma^n \times \Sigma^n \, | \, \forall I \in \mathcal{L}(S) \, . \, if \, u \, \models_{\mathcal{V}} \, I \, then \, v \models_{\mathcal{V}} I \right\rbrace
\end{equation}

Because regular languages are closed under complement, we can define $\overline{C}$ as followings:
\begin{lemma}\label{lemma:abstractly_reachable2}
    
\end{lemma}
\textit{
    Let $\mathcal{R} = (\Sigma, I, T)$ be an RTS, $\langle \Gamma, \mathcal{V} \rangle$ an interpretation, 
    and S a NFA over the alphabet $\Gamma$. Then there exists a  $\Sigma-\Sigma-transducer$ $\overline{C}$ such that
}
\begin{equation}
    [[\overline{C}]] = \left\lbrace \langle u,v\rangle \in \bigcup_{n \geq 0} \Sigma^n \times \Sigma^n \, | \, \exists I \in \mathcal{L}(S) \, . \, if \, u \, \models_{\mathcal{V}} \, I \, then \, v \not\models_{\mathcal{V}} I \right\rbrace
\end{equation}

Note that, these Lemmas have been proved in the previous work \cite*{Welzel2023InductiveSts}.
We will utilize them later for constructing the equivalent oracle in the implementation \autoref{chapter:implementation}.