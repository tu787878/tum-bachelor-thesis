\chapter{Algorithmic Learning of Finite Automata}\label{chapter:learning_algorithm}
Learning automata is a computational model for solving problems, where an agent learns 
to optimize its behavior by interacting with an unknown environment. 
The agent, also known as a learner, 
observes the feedback from the teacher, updates its internal state, and adjusts its 
actions accordingly. The teacher 
This interaction process between the learner and the teacher 
is the primary mechanism of learning automata.
In the field of automata learning, there are generally two distinct settings: 
active and passive learning. However, in this thesis, we only focus on the active learning.
We do not introduce passive learning here but refer the interested reader to \cite*{clarke2009model}.

\paragraph*{}
This chapter aims to provide a deeper understanding of the process of learning automata, 
including the roles and responsibilities of the teacher and learner.
\section{The Teacher and Learner}\label{section:teacher_learning}
In this learning scenario, the teacher is proficient in the language being taught 
and is responsible for answering any questions posed by the learner. The learner 
is given the opportunity to ask two types of queries - membership and equivalence. 
Membership queries are used to classify a word based on whether it belongs to the 
language being taught or not. Equivalence queries, on the other hand, are used to 
determine whether an assumed automaton is equivalent to the language the teacher has 
in mind. The learning process continues until the teacher answers an equivalence query 
positively.
\paragraph*{Membership oracle} 
The learner provides a word $w \in \Sigma^{*}$, the teacher replies "yes" 
or "no" depending on whether $w \in \mathcal{L}$ or not.
\paragraph*{Equivalent oracle} 
The learner conjectures a regular language, typically given as a DFA $\mathcal{M}$, 
and the teacher checks whether $\mathcal{M}$ is an equivalent description of the target 
language $\mathcal{L}$, otherwise return an counterexample $u \in \Sigma^{*}$ with 
$u \in \mathcal{L}(\mathcal{M}) \Longleftrightarrow u \notin (\mathcal{L})$.
\section{Algorithms}\label{section:learner_learning}
A learning algorithm—often called learner—learns a regular target 
language $\mathcal{L} \subset \Sigma^{*}$ over an a priori fixed alphabet $\Sigma$ by actively querying a teacher.
We apply several of these algorithms in the course of this thesis.
\subsection{L*}
\subsection{NL*}
\subsection{Kearns-Vazirani}
\subsection{Rivest-Schapire}