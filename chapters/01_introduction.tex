% !TeX root = ../main.tex
% Add the above to each chapter to make compiling the PDF easier in some editors.

\chapter{Introduction}\label{chapter:introduction}
Computer systems grow in size and permeate more and more areas of our lives, such as PCs, mobile phone,
ATMs, car's control systems, an so on.
However, as systems become more complex, they become more difficult to protect against mistakes or attacks.
Additionally, bugs have the potential to cause significant economic damage, and in some cases, even pose a threat to human lives, such as Pentium-Bug \cite*{KniesPentiumBug}, Toyota's unintended acceleration \cite*{koopman2014case}, etc.
Thus the reliability and stability of the software verification are of major importance. 
Although simulation and testing can detect bugs, it cannot guarantee their absence.
Furthermore, in reactive systems like operating systems, servers, and ATMs, when no function is being computed, termination is usually undesirable. 
For this reason, we are interested in \textit{property checking} or \textit{model checking}. 

\paragraph*{}
Model Checking can take on various forms, and one of these is Regular Model Checking (RMC) \cite*{bouajjani2000regular},
where finite automata are used to represent the program that is being verified.
It has been widely used in various real-world applications, ranging from adaptive 
model checking to solving real-world problems (\cite{faster}, \cite{survey}).
% In RMC, sets of words represent system states and regular relations between words represent transitions of the system.
RMC is a technique commonly employed to ensure the safety of a system by verifying safety properties.
To be more specific, our goal is to confirm that a given program cannot execute 
in a way that starts from a set of initial configurations and leads to a set 
of bad configurations. 
In other words, these bad configurations should be not reached during the program's execution. 
However, this is an undecidable question in general and tools for Regular Model Checking 
are necessarily incomplete.
A semi-decision procedure was proposed in \cite*{Welzel2023InductiveSts} that utilizes 
$\textit{inductive statements}$ to ensure that no undesired configuration can be reached 
from any initial configuration. 
This means there is at least one inductive statement that is satisfied by the initial configuration but not the undesired one.

\paragraph*{}
Over the past decade, there has been a significant increase in the study of automata learning.
This field has produced numerous successful applications, such as 
computational biology, data mining, robotics, automatic verification, and even the analysis of music. 
For an extensive survey, please refer to \cite*{de2005bibliographical}.
This thesis is inspired by successful automata learning techniques and aims to expand its application to verify safety properties for RMC. 
One type of automata learning is active learning, in which the learner attempts to acquire a regular language from a teacher 
who has complete knowledge of the target language.
The learner interacts with the teacher via two type of queries: membership queries and equivalence queries.
The first question pertains to whether a word is part of the target language, 
while the second question is whether a hypothesized automaton can recognize the target language.
Once we know how to implement the Teacher to answer the oracles, 
applying different learning algorithms to obtain a set of inductive statements that are capable of establishing a given safety property is now simple.
% These inductive statements' language serves as proof of the property's correctness.
\paragraph*{}
The purpose of this thesis is to employ various algorithms such as $L^*$, $NL^*$, Kearns-Vazirani, Rivest-Schapire
to learn the set of inductive statements of a system. 
Additionally, we gather and analyze empirical data on the performance of these learning algorithms.


\paragraph*{Structure of the thesis}
\paragraph*{}

The structure of this thesis is as follows.
In \autoref{chapter:preliminaries} of this thesis, we will establish the notations and definitions that will be used throughout the rest of the paper.
In \autoref{chapter:inductive_statement}, we will thoroughly explain the
\textit{regular transition system} and \textit{inductive statements} as an approach 
for checking the safety properties of \textit{RMC}. 
Subsequently, \autoref{chapter:learning_algorithm} gives a general introduction
to active learning algorithms and their oracles.
Furthermore, we will introduce some active learning algorithms that used 
for our experiments.
\autoref{chapter:implementation} will investigate the C++-implemented program to
learn a set of inductive statements from systems called \textit{dodo}. 
% The program uses not only the \textit{Angluin's algorithm $L^*$}, but also 
% the \textit{$NL^*$}, \textit{Kearns-Vaziran} and \textit{Rivest-Schapi}.
% After learning process, it visualizes the graphs that can evaluate the \textit{efficiency} and \textit{effectiveness} of these algorithms.
Finally, we will summarize and assess our experiment results in \autoref{chapter:experiment} 
and conclude the thesis with \autoref{chapter:conclusion}.