\chapter{Implementation}\label{chapter:implementation}

We use automata learning algorithms to solve regular 
model checking problems and generate inductive statements for the parameterized systems.

\section{Membership oracle}
On a membership oracle, the learner provides a statement and asks the teacher if 
this statement whether inductive or not. As we described in Definition \ref{definition:inductive_statements}, 
a statement $I$ is \textit{inductive} if, for any transition $v \rightsquigarrow u$
where $u$ satisfies $I$, $u$ also satisfies the statement.
Oone can implement the Membership Oracle by checking the acceptance 
of $\mathcal{M}$, where $\mathcal{M}$ is an automaton for 
$\overline{Inductive_{\mathcal{V}}(\mathcal{R})}$ and negating the answer (Algorithm \ref{alg:membership}).
The $\overline{Inductive_{\mathcal{V}}(\mathcal{R})}$ is defined by:

\begin{equation}\label{eq:non_inductive}
\overline{Inductive_{\mathcal{V}}(\mathcal{R})} = \lbrace I \in \Gamma^* \, | \, \exists u
\rightsquigarrow_\mathcal{T} w \, . \, u \models \, I \, and \, w \, \not\models I\rbrace
\end{equation}

Let $\mathcal{T} =  \langle P, \Sigma \times \Sigma, \Delta, p_0, E \rangle$ 
is a transducer and  $\mathcal{V} =  \langle Q, \Sigma \times \Gamma, \delta, q_0, F \rangle$ is 
an interpretation. The automaton of $\overline{Inductive_{\mathcal{V}}(\mathcal{R})}$ 
is defined by $\langle Q \times P \times Q, \Gamma, \triangle, \langle q_o,  p_0, q_o \rangle, 
E \times F \times (Q \setminus F) \rangle$ where

\begin{equation*}
    \triangle(\langle q_1, p, q_2 \rangle, I) =  \exists \langle \sigma_1, \sigma_2 \rangle \in \Sigma \times \Sigma. \,\,\,
    (\delta(q_1, \langle \sigma_1, I \rangle) ,\, \Delta(p, \langle \sigma_1, \sigma_2 \rangle) ,\, \delta(q_2, \langle \sigma_2, I \rangle))
\end{equation*}
The states that are accepted by this automaton when each its parts are satified: 
\begin{align*} 
    \delta(q_1, \langle \sigma_1, I \rangle) \in  F \\
    \Delta(p, \langle \sigma_1, \sigma_2 \rangle) \in E \\
    \delta(q_2, \langle \sigma_2, I \rangle) \notin  F
\end{align*}
For every pair of initial word and its reached word through the transducer.
Where the initial word is satified by the statement I, the reached word is not.
From \ref{eq:non_inductive} it can guarantee that all statements, that are acepted by this automaton, are non-inductive.
\begin{algorithm}
\caption{Membership oracle}\label{alg:membership}
\textbf{Input: } \textit{Statement} $\mathcal{I}$ 

\textbf{Output: } \textit{True} or \textit{False}

begin
\begin{algorithmic}[1]
    \State $\mathcal{M} \gets getAutomaton(\overline{Inductive_{\mathcal{V}}(\mathcal{R})})$
    \If{$\mathcal{I} \in \mathcal{L}(\mathcal{M}) $}
        \State return \textit{false};
    \Else
        \State return \textit{true};
    \EndIf
\end{algorithmic}
end
\end{algorithm}
\section{Equivalent oracle}
When the learner provides a conjecture, the teacher checks if it satisfies
the safety property. It it does, the teacher return \textit{true}. Otherwise, the learner
receives a counter example and repeates the proocess.

\begin{algorithm}
    \caption{Equivalent oracle}\label{alg:equivalence}
    \textbf{Input: } \textit{Statement} $\mathcal{I}$ 

    \textbf{Output: } \textit{True}, X, or $I \in \Gamma^*$
    
    begin
    \begin{algorithmic}[1]
        \State $\mathcal{M} \gets getAutomaton(\overline{Inductive_{\mathcal{V}}(\mathcal{R})})$
        \If {$\mathcal{L}(\mathcal{H}) \cap \mathcal{L}(\mathcal{M}) \neq \emptyset$} \Comment{Make sure that all statements are inductive}
            \State return $I \in \mathcal{L}(\mathcal{H}) \cap \mathcal{L}(\mathcal{M})$
        \EndIf

        \State $\mathcal{D} \leftarrow getAutomatonFor(\mathcal{L}(\mathcal{I}) \circ \overset{\mathcal{L}(\mathcal{H})}{\Rightarrow} \circ \mathcal{L}(\mathcal{B})) $
        \Comment{Check safety property}

        \If {$\mathcal{D} = \emptyset$}
            \State return True
        \EndIf

        \State $[\substack{u_1 \\ v_1}] \dots [\substack{u_n \\ v_n}] \leftarrow getWordFrom(\mathcal{L}(\mathcal{D}))$
        \State $I = disprove([\substack{u_1 \\ v_1}] \dots [\substack{u_n \\ v_n}])$
        
        \If {$I = null$}
            \State return X \Comment{throw exception when can not disprove}
        \EndIf
        \State return I
    \end{algorithmic}
    end
\end{algorithm}

Firstly, we ensure the automaton only accepts 
inductive statements. 
We intersect the automaton of $\overline{Inductive_{\mathcal{V}}(\mathcal{R})}$ with the hypothesis.
If there exists any non-inductive statement in the hypothesis,
we return it as counterexample.

Since hypothesis $\mathcal{H}$ does not accept any 
non-inductive statement, we will check with the safety problems
to make sure that the hypothesis strong enough.
Intuitively, the automaton $\mathcal{D}$ in Algorithm \ref*{alg:membership} contains all pairs from initial and bad words,
which is induced by the inductive statements $\mathcal{L}(\mathcal{H})$.
In other words, the safety property is that the inductive statements should not induce the
initial and bad word. We return true and terminates the algorithm if $\mathcal{L}(\mathcal{D}) = \emptyset$.
Otherwise we obtain a counterexample 
$\langle u_1 \dots u_n, \, v_1 \dots v_n \rangle \in \mathcal{L}(\mathcal{D})$.
Intuitively, we can see that $\mathcal{D}$ is the intersection of
the automaton $[[C]]$ (Lemma \ref{lemma:abstractly_reachable}) and $\mathcal{I} \circ  \mathcal{B}$.
Because regular languages are closed under complement, $[[\overline{C}]]$ is defined with:
\begin{equation}
    [[\overline{C}]] = \left\lbrace \langle u,v\rangle \in \bigcup_{n \geq 0} \Sigma^n \times \Sigma^n \, | \, \exists I \in \mathcal{L}(S) \, . \, if \, u \, \models_{\mathcal{V}} \, I \, then \, v \not\models_{\mathcal{V}} I \right\rbrace
\end{equation}
Since computing $[[\overline{C}]]$ is effectively, we will construct the automaton for $[[\overline{C}]]$ and complement it.
Let $S =  \langle P, \Gamma, \Delta, p_0, E \rangle$ 
is a transducer and  $\mathcal{V} =  \langle Q, \Sigma \times \Gamma, \delta, q_0, F \rangle$ is 
an interpretation. The automaton of $\overline{[[C]]}$ 
is defined by $\langle Q \times P \times Q, \Sigma \times \Sigma, \triangle, \langle q_o,  p_0, q_o \rangle, 
E \times F \times (Q \setminus F) \rangle$ where
\begin{equation*}
    \triangle(\langle q_1, p, q_2 \rangle, \langle \sigma_1, \sigma_2 \rangle) =  \exists I \in \Gamma. \,\,\,
    (\delta(q_1, \langle \sigma_1, I \rangle) ,\, \Delta(p, I) ,\, \delta(q_2, \langle \sigma_2, I \rangle))
\end{equation*}
The states that are accepted by this automaton when each its parts are satified: 
\begin{align*} 
    \delta(q_1, \langle \sigma_1, I \rangle) \in  F \\
    \Delta(p, I) \in E \\
    \delta(q_2, \langle \sigma_2, I \rangle) \notin  F
\end{align*}

\section{The word problem}
We are now trying to locate a counterexample $I \in Inductive_{\mathcal{V}}(\mathcal{R})$ that disproves the given hypothesis. 
This is done to ensure that $u_1 \dots u_n \models_v I$
and $v_1 \dots v_n \not\models I$, our inductive statements will no longer induce this pair.
We can also call I an active counterexample since I is in the target language but was missing in 
the candicate language. 
It gives rise to the question whether $I \in Inductive_{\mathcal{V}}(\mathcal{R})$ exists such that $u_1 \dots u_n \models_v I$
and $v_1 \dots v_n \not\models I$.
It was previously proven by \cite{latex} that this problem is in NP.
Moreover, since SAT problem is NP-hard, it can be reduced to SAT.

\paragraph*{Flow interpretation}
In this section, we will extract separating inductive statements using CNF-SAT. 
The entire formular is a conjunction (AND) of clauses, where each clause is a disjunction (OR) of literals.
The idea is, we assign each combination of an alphabet $\sigma \in \Sigma$ and the position 
a variable. 
In other words, a pair $\langle \sigma, i \rangle$ is a literal which assigns to a variable.
The variables have two value: \textit{true} and \textit{false}.
Therefore, $\sigma$ is a part of $I_i$ if and only if the  model value of the literal $\langle \sigma, i \rangle$ is true.
Firstly, we introduce the helper function
\begin{equation*}
    ExactlyOne(V) = \bigvee_{v \in V} v \wedge \bigwedge_{v,v'\in V: v \neq v'} \lnot (v \wedge v')
\end{equation*}
Intuitively, it generates a set of clauses that ensure that exactly one of the 
literals evaluate to \textit{true}.
Recall that a statement is not inductive if there exists one transition
$[\substack{u_1 \\ v_1}] \dots [\substack{u_n \\ v_n}]$ that is accepted by transducer $\mathcal{T}$
for which holds that $x_1 \dots x_n \models I_1 \dots I_n$ and $y_1 \dots y_n \not\models_{\mathcal{V}_{flow}} I_1 \dots I_n$.
Formally, we add these clauses to the formular:
\begin{equation}\label{qu:exactlyOne}
    ExactlyOne(\bigcup_{1 \leq i \leq n} \{ \langle u_i, i \rangle \})
\end{equation}
and 
\begin{equation}\label{qu:notExactlyOne}
    \lnot ExactlyOne(\bigcup_{1 \leq i \leq n} \{ \langle v_i, i \rangle \})
\end{equation}
The clause \ref{qu:exactlyOne} guarantee that there is exactly one
$1 \leq i \leq n$ such that $x_i \in I_i$. 
The clause \ref{qu:notExactlyOne} guarantee that either there is no 
or more than one $1 \leq i \leq n$ such that $x_i \in I_i$.
Semantically, we define a state $\langle l,q,k \rangle \in \{0,1\} \times Q_{\mathcal{T}} \times \{0,1,2\}$ 
corresponds to the observation that one
can reach the state q of $\mathcal{T}$ with a word $[\substack{u_1 \\ v_1}] \dots [\substack{u_n \\ v_n}]$
such that there are k many indices i where $x_i \in I_i$, on the other hand,
there are l many indices j where $y_j \in I_j$.
We need to ensure that each pair $[x,y]$we consider is accepted by the transduce \textit{transducer} $\mathcal{T}$.
Additionally, in the final step should not result in the same configuration for both the source and target. 
To achieve this, we encode the state product as literals and define the proposition formula as follows:

\begin{multline}
    \bigvee_{q_0 \in Q_0^{\mathcal{T}}} \langle \langle 0, q_0, 0 \rangle , 0 \rangle \wedge \lnot \bigvee_{f \in F_{\mathcal{T}}} \langle \langle 1,f,0 \rangle,n \rangle \vee \langle \langle 1,f,2\rangle,n \rangle\\
    \wedge \bigwedge_{1 \leq i \leq n, \,\, \left\langle q, \left[\substack{x \\ y}\right],p \right\rangle \in \Delta_{\mathcal{T}}}
    \begin{pmatrix}
        \langle \langle 0,q,0 \rangle,i \rangle \wedge \langle x, i+1 \rangle \wedge \langle y, i+1 \rangle \implies \langle \langle 1,p,1 \rangle,i+1 \rangle \\
        \wedge \langle \langle 0,q,1 \rangle,i \rangle \wedge \langle x, i+1 \rangle \wedge \langle y, i+1 \rangle \implies \langle \langle 1,p,2 \rangle,i+1 \rangle \\
        \wedge \langle \langle 0,q,2 \rangle,i \rangle \wedge \langle x, i+1 \rangle \wedge \langle y, i+1 \rangle \implies \langle \langle 1,p,2 \rangle,i+1 \rangle \\
        \wedge \bigwedge_{k \in \{0,1\}}
        \begin{pmatrix}
            \langle \langle k,q,0 \rangle,i \rangle \wedge \lnot \langle x, i+1 \rangle \wedge \langle y, i+1 \rangle \implies \langle \langle k,p,1 \rangle,i+1 \rangle \\
        \wedge \langle \langle k,q,1 \rangle,i \rangle \wedge \lnot \langle x, i+1 \rangle \wedge \langle y, i+1 \rangle \implies \langle \langle k,p,2 \rangle,i+1 \rangle \\
        \wedge \langle \langle k,q,2 \rangle,i \rangle \wedge \lnot \langle x, i+1 \rangle \wedge \langle y, i+1 \rangle \implies \langle \langle k,p,2 \rangle,i+1 \rangle
        \end{pmatrix} \\
        \wedge \bigwedge_{l \in \{0,1,2\}}
        \begin{pmatrix}
            \langle \langle 0,q,l \rangle,i \rangle \wedge \langle x, i+1 \rangle \wedge \lnot \langle y, i+1 \rangle \implies \langle \langle 1,p,l \rangle,i+1 \rangle \\
        \wedge \langle \langle 0,q,l \rangle,i \rangle \wedge \langle x, i+1 \rangle \wedge \lnot \langle y, i+1 \rangle \implies \langle \langle 1,p,l \rangle,i+1 \rangle
        \end{pmatrix} \\
        \wedge \bigwedge_{k \in \{0,1\} \,\, l \in \{0,1,2\} }
        \begin{pmatrix}
            \langle \langle k,q,l \rangle,i \rangle \wedge \lnot \langle x, i+1 \rangle \wedge \lnot \langle y, i+1 \rangle \implies \langle \langle k,p,l \rangle,i+1 \rangle
        \end{pmatrix}
    \end{pmatrix}
\end{multline}

To solve the (CNF-)SAT problem we need to convert the entire of 
logical formular in the CNF form, 
which can be achieved by applying DeMorgan's laws \cite{enwiki:1184283195}.


% \paragraph*{SAT-Solver for trap and siphon interpretation}
% We are only considering trap interpretation since the siphon is analog.
% For a word $w$ to satisfy the statement I, there must be at least one indice
% i where $x_i \in I_i$.
% Unlike the state product used for Flow interpretation, k is now wither \textit{true} or \textit{false}, 
% depeding on whether there is at least one indice i
% where $y_i \in I_i$.
% We do not consider the the cases for l because l is always \textit{true}
% which ensure that x satisfy the statement.
% Therefore, we change the format of the state product to 
% $Q_{\mathcal{T}} \times \{true, false\}$. 
% Additionally, we define the rules for initial and final states sequentially: 
% \begin{equation*}
%     \bigvee_{q_0 \in Q_0^{\mathcal{T}}} \langle \langle q_0, false\rangle , 0 \rangle \wedge \lnot \bigvee_{f \in F_{\mathcal{T}}} \langle \langle f,true \rangle,n \rangle
% \end{equation*}
% The transitions should be as followings:
% \begin{multline}
%     \wedge \bigwedge_{1 \leq i \leq n, \,\, \left\langle q, \left[\substack{x \\ y}\right],p \right\rangle \in \Delta_{\mathcal{T}}}
%     \begin{pmatrix}
%         \langle \langle q,false \rangle,i \rangle \wedge \lnot \langle x, i+1 \rangle \wedge \lnot \langle y, i+1 \rangle \implies \langle \langle p,false \rangle,i+1 \rangle \\
%         \wedge \langle \langle q,false \rangle,i \rangle \wedge \langle x, i+1 \rangle \wedge \lnot \langle y, i+1 \rangle \implies \langle \langle p,true \rangle,i+1 \rangle \\
%         \wedge \langle \langle q,true \rangle,i \rangle \wedge \lnot \langle y, i+1 \rangle \implies \langle \langle p,true \rangle,i+1 \rangle \\
%     \end{pmatrix}
% \end{multline}


\paragraph{A polynomial time algorithm for the word problem for $\mathcal{V}_{trap}$ and $\mathcal{V}_{siphon}$}
Again, we focus on $\mathcal{V}_{trap}$ since the arguments for $\mathcal{V}_{siphon}$ are analogous.
Pseudocode \ref{algorihtm:disprove} demonstrates how to find the separating 
statement for trap interpretation within polynomial time.
We begin with the statment $I = \Sigma \backslash \{y_1\} \dots \Sigma \backslash \{y_n\}$.
If a transition $\left[\substack{x_1 \\ y_1}\right] \dots \left[\substack{x_n \\ y_n}\right]$ exists such that 
$x_1 \dots x_n$ satisfies the current statement and $y_1 \dots y_n$ does not,
then remove $x_i$ from the i-th letter of the statement for all $1 \leq i \leq n$.
To prove that this approach can be computed in polynomial time, refers to \cite{latex}.

\begin{algorithm}
    \caption{Disprove}\label{algorihtm:disprove}
    \textbf{Input: } $\left[\substack{x_1 \\ y_1}\right] \dots \left[\substack{x_n \\ y_n}\right]$ and transducer $\mathcal{T}$

    \textbf{Output: } Inductive statement I
    
    begin
    \begin{algorithmic}[1]
        \For{$i = 1; \, i \leq n; \, i = i+1 $} 
            \State {$I_i = \Sigma \backslash \{y_i\}$} 
        \EndFor
        \While{$\langle v, I \rangle \in \mathcal{L}(\mathcal{V}_{trap})$}
            \If {$\exists \left[\substack{x_1 \\ y_1}\right] \dots \left[\substack{x_n \\ y_n}\right]$ where $u_1 \dots u_n \models I$ and $v_1 \dots v_n \not\models I$} 
                \For{$i = 1; \, i \leq n; \, i = i+1 $}     
                    \State {$I_i = I_i \backslash \{u_i\}$} 
                \EndFor
            \Else
                \State return I
            \EndIf
        \EndWhile
        \State return $\emptyset$
    \end{algorithmic}
    end
\end{algorithm}
The language we are learning has a much exponentially larger alphabet than the RTS. 
However, \textit{Libalf} - the software we are using - allows us to start the learning process with a smaller alphabet, 
which we can expand later if we need to. 
So, we begin with an empty alphabet and gradually add letters from $2^\Sigma$.
