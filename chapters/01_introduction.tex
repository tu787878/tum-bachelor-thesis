% !TeX root = ../main.tex
% Add the above to each chapter to make compiling the PDF easier in some editors.

\chapter{Introduction}\label{chapter:introduction}
% Introduce software systems and testing problems.
% Then connect to model checking
As software systems grow in size and permeate more and more areas of our lives.
Individuals and organizations use the majority of software in their systems. Thus the 
reliability and stability of the software testing are of major importance. Simulation 
and testing can detect bugs but not prove their absence. In such reactive 
systems, when no function is being computed, termination is usually undesirable. For 
this reason, we are interested in \textit{property checking} or \textit{model checking}. 
It has found a wide range of applications spanning from adaptive model checking.
\paragraph*{}
% Introduce model checking, then regular model checking, then inductive statements
We here consider the verification of safety properties similar to the original 
Regular Model Checking framework, where a program is represented using symbols 
and finite automata.
To be more specific, our goal is to confirm that a given program cannot execute 
in a way that starts from a set of initial configurations ($\mathcal{I}$) and leads to a group 
of dedicated bad configurations ($\mathcal{B}$). These bad configurations represent conditions 
that should not happen during the program's execution. However, this is an 
undecidable question in general and tools for Regular Model Checking 
still need to be completed.
A solution to this problem was proposed in \cite*{clarke2009model}, which utilizes 
$\textit{inductive statements}$ to ensure that no undesired configuration can be reached 
from any initial configuration. This means that for every pair of initial 
and undesired configurations, there is at least one inductive statement that 
is satisfied by the initial configuration but not by the undesired one. 
By doing this, it can be concluded that no undesired configuration can be reached.
\paragraph*{}
% Introduce learning for inductive statements
Over the past decade, there has been a significant increase in the study of automata learning.
This field has produced numerous successful applications, such as pattern and natural language recognition, 
computational biology, data mining, robotics, automatic verification, and even the analysis of music.
One can use autoamta learning to acquire a set of inductive statements that 
are powerful enough to establish a given safety property. The language of these
inductive statements serves as proof of the property's correctness.
The purpose of this thesis is to collect and analyze empirical data on 
the performance of learning algorithms such as L*, NL*, Kearns-Vazirani, Rivest-Schapire.


\paragraph*{Structure of the thesis}
\paragraph*{}

In this thesis, we begin with \autoref{chapter:preliminaries} by fixing notations and 
definitions used throughout the thesis.
In \autoref{chapter:inductive_statement}, we will thoroughly explain the
\textit{Regular transition system} and \textit{Inductive statements} as an approach 
for checking the safety properties of \textit{model checking}. 
Subsequently, \autoref{chapter:learning_algorithm} gives a general introduction
to active learning algorithm and their oracles.
Furthermore, we will introduce some active learning algorithms used 
for our experiments.
\autoref{chapter:implementation} will investigate the C++-implemented programm to
learn a set of inductive statements from a system configuation called \textit{dodo}. 
The programm uses not only the \textit{Angluin's algorithm $L^*$}, but also 
the \textit{$NL^*$}, \textit{Kearns-Vaziran} and \textit{Rivest-Schapi}.
After learning process, it visualizes the graphs that can evaluate 
the \textit{efficiency} and \textit{effectiveness} of these algorithms.
Finally, we will summarize and assess our experment results in \autoref{chapter:experiment} 
and conclude the thesis with \autoref{chapter:conclusion}.