\chapter{Preliminaries}\label{chapter:preliminaries}
% basic
In this chapter, we introduce some basic notions and definitioms that we use throughout this thesis.

\section*{Words and Languages}
An \textit{alphabet} $\Sigma$ is a finite set of symbols.
A \textit{word} $u = a_1 \dots a_n$ is a finite sequence of symbols $a_i \in \Sigma$ for $i \in \lbrace 1, \dots, n\rbrace$. 
$\Sigma^*$ denotes the set of all words over an alphaber $\Sigma$.
\textit{Regular languages} are those which can be identified by a finite state automaton \cite*{enwiki:1190424473}.

\section*{Finite automata}
We classify automata into two categories: deterministic and non-deterministic, 
in order to identify regular languages consisting of finite words.
\begin{theo}[Deterministic finite automaton (DFA)]{definition:dfa}
    \textit{
    A DFA is a quintuple $\mathcal{M} = (Q, q_0, \Sigma, \delta, F)$ where $Q$ is a finite set of states with 
    a initial state $q_0 \in Q$. A set of input symbols called the alphabet $\Sigma$.
    A transition $\delta: Q \times \Sigma \rightarrow  Q$ and a set of final states $F$.
    Let $w=a_1a_2...a_n$ be a string over the alphabet $\Sigma$. The automaton $\mathcal{M}$ 
    accepts $w$ if a sequence of states, $r_0, r_1,...r_n$ exist in $Q$:
    \begin{itemize}
        \item $r_0=q_0$
        \item $r_{i+1} = \delta(r_i, a_{i+1}),$ for $i = 0,...,n-1$
        \item $r_n \in F$
    \end{itemize}
    }
\end{theo}

\begin{theo}[Nondeterministic finite automaton (NFA)]{definition:nfa}
    \textit{
        A NFA is a quintuple $\mathcal{N} = (Q, q_0, \Sigma, \Delta, F)$ where $Q$, $\Sigma$ and $F$ 
        are as for a DFA.
        Let $w=a_1a_2...a_n$ be a string over the alphabet $\Sigma$. The automaton $\mathcal{N}$ 
        accepts $w$ if a sequence of states, $r_0, r_1,...r_n$ exist in $Q$:
        \begin{itemize}
            \item $r_0=q_0$
            \item $r_{i+1} \in \Delta(r_i, a_{i+1}),$ for $i = 0,...,n-1$
            \item $r_n \in F$
        \end{itemize}
        }
\end{theo}

\section*{Token passing algorithm}\label{example:token-passing}
We will provide a simple example to demonstrate how systems are 
modelled in \textit{regular transition system}.
The \textit{token passing} system comprises a linear array of agents where the 
agent holds a token, and in each step, the current agent can pass 
the token to its right neighbour. We choose to represent the agent that holds 
the token as the letter t and the agents that do not hold the token
as the letter n.

% Let us examine how this algorithm is applied in a real-world scenario. 
% Imagine a conveyor belt buffet restaurant where a big plate of thinly sliced beef 
% starts at the first table and moves to the next table every 5 seconds. 
% This process continues until the plate reaches the last table, where it stops. 
% In this case, the token represents the plate of beef and the tables are the agents.
% We want to avoid any potential problems that may arise. 
% For instance, if there are no plates on the conveyor, customers may become 
% frustrated. On the other hand, if there are too many plates of beef on the 
% conveyor at once, the boss may worry about revenue.