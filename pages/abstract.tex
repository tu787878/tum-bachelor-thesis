\chapter{\abstractname}

Regular Model Checking, as proposed in the works of \cite*{bouajjani2000regular}, is a 
widely used paradigm for verifying parameterized and infinite-state systems that 
occur naturally when a program uses queries, stacks or intergers, etc.



%TODO: Abstract
This thesis aims to investigate automata learning,
which includes techniques that extract finite automata 
from external data sources in the domains of verification and synthesis.
We are considering an application scenario that is particularly well-suited for Regular Model Checking, 
where the \textit{regular transition system} models the parameterized systems.

Firtsly, we present an approach for regular model checking that uses inductive statements to over-approximate all reachable states.
A statement is inductive if it only relates a state satisfying $\phi$ with states that also satisfy $\phi$.
We demonstrate how the statements are encoded and their \textit{interpretations} defined, 
which helps in understanding the encoded statements.

We will discuss the primary mechanism of learning automata, which involves the use of 
membership queries and equivalent queries. 
During the learning process, the Teacher and the Learner interact with each other. 
The Teacher has knowledge of the target language, 
while the Learner has the opportunity to ask two types of queries: membership and equivalence queries.
Additionally, we will cover four active learning algorithms: $L^*$, $NL^*$, Kearns-Vazirani, Rivest-Schapire.

We evaluated the performance of our tool, dodo-cpp, on a set of common examples for parameterized verification, 
and compared the results of these algorithms.
The main findings of this thesis show that: (1.) $\dots$, (2.) $\dots$, (3.) $\dots$ (NEED MORE INFO FROM EXPERIMENT)


