\chapter{Preliminaries}\label{chapter:preliminaries}
% basic
In this chapter, we introduce some basic notions and definitions that we use throughout this thesis.

\section*{Words and Languages}
An \textit{alphabet} $\Sigma$ is a finite set of symbols.
A \textit{word} $u = a_1 \dots a_n$ is a finite sequence of symbols $a_i \in \Sigma$ for $i \in \lbrace 1, \dots, n\rbrace$. 
$\Sigma^*$ denotes the set of all words over an alphabet $\Sigma$.
\textit{Regular languages} are those which can be identified by a finite state automaton.

\section*{Finite automata}
We classify automata into two categories: deterministic and non-deterministic, 
in order to identify regular languages consisting of finite words.
\begin{theo}[Deterministic finite automaton (DFA)]{definition:dfa}
    \textit{
    A DFA is a quintuple $\mathcal{M} = (Q, q_0, \Sigma, \delta, F)$ where $Q$ is a finite set of states with 
    a initial state $q_0 \in Q$. $\Sigma$ is set of alphabet of the automaton.
    We define the transition $\delta: Q \times \Sigma \rightarrow  Q$ and $F$ is a set of final states.
    Let $w=a_1a_2\dots a_n$ be a string over the alphabet $\Sigma$. The automaton $\mathcal{M}$ 
    accepts $w$ if a sequence of states, $r_0, r_1,\dots,r_n$ exist in $Q$:
    \begin{itemize}
        \item $r_0=q_0$
        \item $r_{i+1} = \delta(r_i, a_{i+1}),$ \, for $i = 0,\dots,n-1$
        \item $r_n \in F$
    \end{itemize}
    }
\end{theo}
\newpage

\begin{theo}[Nondeterministic finite automaton (NFA)]{definition:nfa}
    \textit{
        A NFA is a quintuple $\mathcal{N} = (Q, q_0, \Sigma, \Delta, F)$ where $Q$, $q_0$, $\Sigma$ and $F$ 
        are defined similarly to Definition \ref{definition:dfa}.
        Let $w=a_1a_2\dots a_n$ be a string over the alphabet $\Sigma$. The automaton $\mathcal{N}$ 
        accepts $w$ if a sequence of states, $r_0, r_1\dots,r_n$ exist in $Q$:
        \begin{itemize}
            \item $r_0=q_0$
            \item $r_{i+1} \in \Delta(r_i, a_{i+1}),$ \, for $i = 0,\dots,n-1$
            \item $r_n \in F$
        \end{itemize}
        }
\end{theo}

Here are two examples of presenting regular languages using DFA and NFA.
$\mathcal{L} = ab*$ is a regular language that consists of all words over the alphabet $\Sigma = \{a,b\}$, 
which begin with one "a" and are followed by either any number of "b" or nothing at all.

\begin{lem}[DFA for language $\mathcal{L} = ab*$]{lem:dfa}
    \textit{Automaton over the alphabet $\Sigma = \{a,b\}$}.\\
    \begin{tikzpicture}[shorten >=1pt,node distance=2cm,on grid,auto] 
        \node[state,initial] (q_0)   {$q_0$}; 
        \node[state,accepting] (q_1) [right=of q_0] {$q_1$}; 
        \node[state] (q_2) [below=of q_1] {$q_2$}; 
         \path[->] 
         (q_0) edge  node {a} (q_1)
         (q_0) edge  node {b} (q_2)
         (q_1) edge [loop above] node {b} ()
         (q_1) edge  node {a} (q_2)
         (q_2) edge [loop below] node {a,b} ();
     \end{tikzpicture}
\end{lem}

\newpage
\begin{lem}[NFA for language $\mathcal{L} = ab*$]{lem:nfa}
    \textit{Automaton over the alphabet $\Sigma = \{a,b\}$}.\\
    \begin{tikzpicture}[shorten >=1pt,node distance=2cm,on grid,auto] 
        \node[state,initial] (q_0)   {$q_0$}; 
        \node[state,accepting] (q_1) [right=of q_0] {$q_1$}; 
         \path[->] 
         (q_0) edge  node {a} (q_1)
         (q_1) edge [loop above] node {b} ();
     \end{tikzpicture}
\end{lem}


% Let us examine how this algorithm is applied in a real-world scenario. 
% Imagine a conveyor belt buffet restaurant where a big plate of thinly sliced beef 
% starts at the first table and moves to the next table every 5 seconds. 
% This process continues until the plate reaches the last table, where it stops. 
% In this case, the token represents the plate of beef and the tables are the agents.
% We want to avoid any potential problems that may arise. 
% For instance, if there are no plates on the conveyor, customers may become 
% frustrated. On the other hand, if there are too many plates of beef on the 
% conveyor at once, the boss may worry about revenue.