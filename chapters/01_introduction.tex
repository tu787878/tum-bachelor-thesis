% !TeX root = ../main.tex
% Add the above to each chapter to make compiling the PDF easier in some editors.

\chapter{Introduction}\label{chapter:introduction}
% Introduce software systems and testing problems.
% Then connect to model checking
As software systems grow in size and permeate more and more areas of our lives.
Individuals and organizations use the majority of software in their systems. Thus the 
reliability and stability of the software testing are of major importance. Simulation 
and testing can detect bugs but not prove their absence. Such reactive 
systems, when no function is being computed, termination is usually undesirable. For 
this reason, we are interested in \textit{property checking} or \textit{model checking}. 
It has found a wide range of applications spanning from adaptive model checking.
\paragraph*{}
% Introduce model checking, then regular model checking, then inductive statements
\textit{Model checking} is a powerful technique for automatic verification of finite state 
concurrent systems \cite{clarke2009model}. 
In this thesis, we only focus on \textit{regular model checking}, which is a important 
framework for infinite state model-checking. The model is typically a regular 
transition system. It describes the potential behavior of discrete systems by representing 
it in finite state automaton. In \cite{clarke2009model} the author introduce an approach, 
that using the inductive statements for the regular transition for checking 
safety conditions. ``Statement $\psi$ is inductive if the transition relation only relates 
a state $\nu$ satisfying $\psi$ with states that also satisfy $\psi$. Thus, the set of 
all states that satisfy $\psi$ over-approximates the set of all states reachable from $\nu$''.

\paragraph*{}
% Introduce learning for inductive statements
Based on automata learning, one can learn a set of inductive statements that are powerful 
enough to establish a given safety property.
The learned language of inductive statements is a certificate of the correctness of the 
property. The purpose of this thesis is to collect and analyze empirical data on 
the performance of the learning algorithms such as L*, NL*, Kearns-Vazirani, Rivest-Schapire.


\paragraph*{Structure of the thesis}
\paragraph*{}

In the first part of our thesis, we consider the regular transition system and 
the set of all inductive statements. In the second part of the thesis,