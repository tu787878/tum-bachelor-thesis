% !TeX root = ../main.tex
% Add the above to each chapter to make compiling the PDF easier in some editors.

\chapter{Introduction}\label{chapter:introduction}
% Introduce software systems and testing problems.
% Then connect to model checking
As software systems grow in size and permeate more and more areas of our lives.
Individuals and organizations use the majority of software in their systems. Thus the 
reliability and stability of the software testing are of major importance. Simulation 
and testing can detect bugs but not prove their absence. Such reactive 
systems, when no function is being computed, termination is usually undesirable. For 
this reason, we are interested in \textit{property checking} or \textit{model checking}. 
It has found a wide range of applications spanning from adaptive model checking.
\paragraph*{}
% Introduce model checking, then regular model checking, then inductive statements
We here consider the verification of safety properties similar to the original 
Regular Model Checking framework, where a program is represented using symbols 
and finite automata.
To be more specific, our goal is to confirm that a given program cannot execute 
in a way that starts from a set of initial configurations ($\mathcal{I}$) and leads to a group 
of dedicated bad configurations ($\mathcal{B}$). These bad configurations represent conditions 
that should not happen during the program's execution. However, this is an 
undecidable question in general and tools for Regular Model Checking are 
necessarily incomplete. 
A solution to this problem was proposed in \cite*{clarke2009model}, which utilizes 
$\textit{inductive statements}$ to ensure that no undesired configuration can be reached 
from any initial configuration. This means that for every pair of initial 
and undesired configurations, there is at least one inductive statement that 
is satisfied by the initial configuration but not by the undesired one. 
By doing this, it can be concluded that no undesired configuration can be reached.
\paragraph*{}
% Introduce learning for inductive statements
Based on automata learning, one can learn a set of inductive statements that are powerful 
enough to establish a given safety property.
The learned language of inductive statements is a certificate of the correctness of the 
property. The purpose of this thesis is to collect and analyze empirical data on 
the performance of learning algorithms such as L*, NL*, Kearns-Vazirani, Rivest-Schapire.


\paragraph*{Structure of the thesis}
\paragraph*{}

In the first part of our thesis, we consider the regular transition system and 
the set of all inductive statements. In the second part of the thesis,