\chapter{\abstractname}

Regular Model Checking is a 
widely used paradigm for verifying parameterized and infinite-state systems that 
occur naturally when a program uses queries, stacks or intergers, and so on.
One of the major challenges in verifying parameterized and infinite-state systems is 
determining the set of states that can be reached from a set of initial states,
which is an undecidable problem and necessary incomplete.

To address this issue, we propose a semi-decision procedure for regular model checking that uses inductive statements to over-approximate all reachable states.
A statement is considered inductive if it only relates a state satisfying $\phi$ with states that also satisfy $\phi$.
We demonstrate how the statements are encoded and their \textit{interpretations} defined, 
which is crucial for understanding the encoded statements.

We discuss the primary mechanism of learning automata, which involves the use of 
membership queries and equivalent queries. 
During the learning process, the Teacher and the Learner interact with each other. 
The Teacher has knowledge of the target language, 
while the Learner has the opportunity to ask two types of queries: membership and equivalence queries.
Additionally, we will cover four active learning algorithms: $L^*$, $NL^*$, Kearns-Vazirani, Rivest-Schapire.

We evaluated the performance of our tool, dodo-cpp, on a set of common examples for parameterized verification, 
and compared the results of these algorithms.
The main findings of this thesis show that: (1.) $\dots$, (2.) $\dots$, (3.) $\dots$ (NEED MORE INFO FROM EXPERIMENT)


