\chapter{Implementation}\label{chapter:implementation}

We apply automata learning algorithms to solve the regular model checking problems
as well as finding an inductive statements for regular transition system.

\paragraph*{Membership query}
On a membership oracle, the learner provides a statement and asks the teacher if 
this statement whether inductive or not. As we described in \autoref{chapter:inductive_statements}, 
a statement $I$ is \textit{inductive} if, for any transition $v \rightsquigarrow u$
where $u$ satisfies $I$, $u$ also satisfies the statement.
REF TO ABOVE DEFINITION.
Therefore, one can simply implement the Membership Oracle by checking the acceptance 
of $\mathcal{M}$ while $\mathcal{M}$ is an automaton for 
$\overline{Inductive_{\mathcal{V}}(\mathcal{R})}$ and negating the answer.
The $\overline{Inductive_{\mathcal{V}}(\mathcal{R})}$ is defined by:
\[ \overline{Inductive_{\mathcal{V}}(\mathcal{R})} = \lbrace I \in \Gamma^* \, | \, \exists u
\rightsquigarrow_\mathcal{T} w \, . \, u \models \, I \, and \, w \, \not\models I\rbrace
\]

Let $\mathcal{T} =  \langle P, \Sigma \times \Sigma, \Delta, p_0, E \rangle$ 
is a transducer and  $\mathcal{V} =  \langle P, \Sigma \times \Gamma, \delta, q_0, F \rangle$ is 
an interpretation. The automaton of $\overline{Inductive_{\mathcal{V}}(\mathcal{R})}$ 
is defined by $\langle Q \times P \times Q, \Gamma, \triangle, \langle q_o,  p_0, q_o \rangle, 
E \times F \times (Q \setminus F) \rangle$ where

\[
    \triangle(\langle q_1, p, q_2 \rangle, I) = \delta()
\]


\begin{algorithm}
\caption{Membership query}\label{alg:membership}
\textbf{Input: } \textit{Statement} $\mathcal{I}$ 

\textbf{Output: } \textit{True} or \textit{False}

begin
\begin{algorithmic}
    \State $\mathcal{M} \gets getAutomaton(\overline{Inductive_{\mathcal{V}}(\mathcal{R})})$
    \If{$\mathcal{I} \in \mathcal{L}(\mathcal{M}) $}
        \State return \textit{false};
    \Else
        \State return \textit{true};
    \EndIf
\end{algorithmic}
end
\end{algorithm}

\begin{algorithm}
    \caption{Equivalent query}\label{alg:cap}
    \begin{algorithmic}
    \Require $n \geq 0$
    \Ensure $y = x^n$
    \State $y \gets 1$
    \State $X \gets x$
    \State $N \gets n$
    \While{$N \neq 0$}
    \If{$N$ is even}
        \State $X \gets X \times X$
        \State $N \gets \frac{N}{2}$  \Comment{This is a comment}
    \ElsIf{$N$ is odd}
        \State $y \gets y \times X$
        \State $N \gets N - 1$
    \EndIf
    \EndWhile
    \end{algorithmic}
    \end{algorithm}