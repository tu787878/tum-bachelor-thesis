\chapter{Inductive statements for regular transition system}\label{chapter:inductive_statement}
\paragraph*{}
In the \textit{Regular Model Checking} framework, program configurations are 
represented as finite words over a pre-determined alphabet $\Sigma$.
The system comprises a series of starting configurations and the transitions.

In this thesis, we will be focusing on \textit{Regular transition system} (RTS) 
which uses regular languages to represent the models of the program.
\section{Regular transition system}\label{section:rts}
The \textit{transducer} will define the behavior of the system, i.e 
it describes how the configurations change in the system.
\begin{definition}\label{definition:transducer}
    Transducer.
\end{definition}
\textit{
    A $\Sigma$-$\Gamma$-\textit{transducer} is an \textit{NFA} 
    $\langle Q, Q_0, \Sigma \times \Gamma, \Delta, F \rangle$,
    it represents a list of pairs $\langle u_1, v_1 \rangle \dots \langle u_n, v_n \rangle$
    where $\langle u_1 \dots u_n, v_1 \dots v_n \rangle \in \bigcup_{n \geq 0} \Sigma^{n} \times \Gamma^{n}$.
}

EXAMPLE FOR TRANSDUCER
\begin{definition}\label{definition:rts}
    Regular transition system (RTS).
\end{definition}
\textit{
   An RTS is a triple $\mathcal{R} = \langle \Sigma, \mathcal{I}, \mathcal{T} \rangle$ where $\Sigma$
   is finite alphabet while $\mathcal{I}$ is an NFA, which represents initial configurations.
   And $\mathcal{T}$ is a $\Sigma$-$\Sigma$-\textit{transducer}.
}

EXAMPLE FOR RTS

\section{Inductive statements}\label{section:inductive_statements}

\begin{definition}\label{definition:interpretation}
    Interpretation.
\end{definition}
\textit{
   For any RTS $\mathcal{R} = \langle \Sigma, \mathcal{I}, \mathcal{T} \rangle$, we call 
   a pair $\langle \Gamma, \mathcal{V} \rangle$ an $\Gamma$-\textit{interpretation}
   where $\Gamma$ is a finite alphabet and $\mathcal{V}$ is a deterministic $\Sigma$-$\Sigma$-\textit{transducer}.
   In the following, we denote $ u \models I $ to indicate $ \langle u, I \rangle \in [[\mathcal{V}]]$.
}

\begin{definition}\label{definition:inductive_statements}
    Inductive statements.
\end{definition}
\textit{
   For any given $\Gamma$-\textit{interpretation} for $\mathcal{R} = \langle \Sigma, \mathcal{I}, \mathcal{T} \rangle$, we define
   \[
   Inductive_{\mathcal{V}}(\mathcal{R}) = \lbrace I \in \Gamma^* | 
   \forall u \rightsquigarrow_\mathcal{T} \, . \, if \langle u, I \rangle
   \in [[\mathcal{V}]] \, then \, \langle v, I \rangle \in [[\mathcal{V}]] \rbrace
   \]
   \[=  \lbrace I \in \Gamma^* | 
   \forall u \rightsquigarrow_\mathcal{T} \, . \, if u \models I
    \, then \, v \models I \rbrace\]
}
\begin{definition}\label{definition:potential_reachability}
    Potential reachability.
\end{definition}
\textit{
   Let $\mathcal{R} = \langle \Sigma, \mathcal{I}, \mathcal{T} \rangle$ be any \textit{RTS}
   and $\langle \Gamma, \mathcal{V} \rangle$ any interpretation.
   We write $u \Rightarrow_{\mathcal{V}} v$ if and only if $u \models_{\mathcal{V}} v$
   for all $I \in target_\mathcal{V}(u) \bigcap Inductive_\mathcal{V}(\mathcal{R})$.
   }
   $\overset{Inductive}{\Longrightarrow_{\mathcal{V}}}$

   \section{Concrete interpretations}