\chapter{Algorithmic Learning of Finite Automata}\label{chapter:learning_algorithm}
Learning automata is a computational model for solving problems, where an agent learns 
to optimize its behavior by interacting with an unknown environment. 
The agent, also known as a learner, 
observes the feedback from the teacher, updates its internal state, and adjusts its 
actions accordingly. 
This interaction process between the $\textit{Learner}$ and the $\textit{Teacher}$ 
is the primary mechanism of learning automata.
In the field of automata learning, there are generally two distinct settings: 
active and passive learning. Passive algorithms are provided with a fixed set of 
examples consisting of strings that the automaton should either accept or reject. 
Active algorithms, unlike passive ones, have the ability to expand the set of examples as needed
by asking further queries.
However, in this thesis, our focus is solely on active learning.
We do not introduce passive learning here but refer the interested reader to \cite*{clarke2009model}.

This chapter intends to offer a comprehensive insight into the 
learning automata process, specifically discussing the roles and 
responsibilities of the Teacher and Learner in \autoref{section:teacher_learning}. 
In \autoref{section:learner_learning}, 
we will introduce multiple active algorithms that we will use for our experiment.
\section{The oracles}\label{section:teacher_learning}
In this learning scenario, the $\textit{Teacher}$ is proficient in the language being taught 
and is responsible for answering any questions posed by the learner. The $\textit{Learner}$ 
is given the opportunity to ask two types of queries - membership and equivalence. 
Membership queries are used to classify a word based on whether it belongs to the 
language being taught or not. Equivalence queries, on the other hand, are used to 
determine whether an assumed automaton is equivalent to the language the $\textit{Teacher}$ has 
in mind. The learning process continues until the $\textit{Teacher}$ answers an equivalence query 
positively.
\paragraph*{Membership oracle} 
The $\textit{Learner}$ provides a word $w \in \Sigma^{*}$, the $\textit{Teacher}$ replies "yes" 
or "no" depending on whether $w \in \mathcal{L}$ or not.
\paragraph*{Equivalent oracle} 
The $\textit{Learner}$ conjectures a regular language, typically given as a DFA $\mathcal{M}$, 
and the $\textit{Teacher}$ checks whether $\mathcal{M}$ is an equivalent description of the target 
language $\mathcal{L}$ and return "yes", otherwise return an counterexample $u \in \Sigma^{*}$ with 
$u \in \mathcal{L}(\mathcal{M}) \Longleftrightarrow u \notin \mathcal{L}$ or 
$u \in \mathcal{L} \Longleftrightarrow u \notin \mathcal{L}(\mathcal{M})$.
\paragraph*{}
On equivalent oracle, the $\textit{Teacher}$ can return a positive counterexample or a negative counterexample \cite{chen2017learning}.
A positive counterexample is a missing word in the conjecture but present in the target.
The negative one is defined symmetic.

Active learning faces the challenge that the runtime of a learning algorithm is 
influenced by the quality of counterexamples. 
If a teacher provides an unnecessarily long counterexample, 
the learner has no option but to process the entire word to make progress.
It is crucial for the $\textit{Teacher}$ to have a clear and specific understanding of the 
correct hypothesis.
Since we know how to implement the $\textit{Teacher}$ to answer the oracles, it is now simple 
to apply different of learning algorithms.
\section{Algorithms}\label{section:learner_learning}
A learning algorithm—often called learner—learns a regular target 
language $\mathcal{L} \subset \Sigma^{*}$ over an a priori fixed alphabet $\Sigma$ by actively querying a teacher.
We apply several of these algorithms in the course of this thesis.
\subsection{L*}
L* learning automata was introduced by Angluin in 1987 \cite{ANGLUIN198787},
also called Angluin's algorithm.
Angluin's algorithm has the ability to learn a 
regular set which is unknown initially, from any \textit{Teachers}.
During the learning process, it stores
information in an observation table 
$\mathcal{O} = (S, E, T)$ where $S \subseteq \Sigma^*$ is a nonempty
\textit{prefix-closed} \footnote{A set of strings S is called prefix-closed if: $uv \in S \implies u \in S$} set, 
a finite \textit{suffix-closed} \footnote{A set of strings S is called suffix-closed if: $uv \in S \implies v \in S$} set E,
and T : $(S \cup S \cdot A) \cdot E \rightarrow \lbrace 0, 1 \rbrace$ 
is a mapping that stores the table entries.
The algorithm maintains $T(u) = 1$ if and only if $u$ is accepted by the target language for all 
$u \in (S \cup S \cdot A) \cdot E$.

Let's take a closer look at the inner workings of the \textit{Angluin's algorithm}.
For each $row(s)$ of the table, where $s \in S$ denotes a function
\[
  f_s : E \rightarrow \lbrace 0, 1 \rbrace \,\, with \,\, f_s(e) = T(s \cdot e)  
\]
The overvation table has two properties: \textit{closed} and \textit{consistent}.
An observation table is called \textit{closed} provided that
for each t in $S \cdot A$ there exists an s in S 
such that row(t) = row(s).
An observation table is called \textit{consistent} provided that 
whenever $s_1$ and $s_2$ are elements of S such that
row($s_1$) = row($s_2$) for all a in A, 
row($s_1 \cdot a$) = row($s_2 \cdot a$).
Once the table is \textit{closed} and \textit{consistent}, we can build a deterministic finite-state acceptor,
which also is called \textit{conjecture}, by using the observation table.
More precisely, \textit{Angluin's algorithm} constructs the DFA 
$\mathcal{H} = (Q, q_0, \Sigma, \delta, F)$ where:
\[
  Q = \lbrace row(s): s \in S \rbrace,
\]
\[
  q_0 = row(\lambda),
\]
\[
  F = \lbrace row(s): s \in S \,\, and \,\, T(s)  = 1\rbrace,
\]
\[
  \delta(row(s),a) = row(s \cdot a).
\]
Basically these two conditions \textit{closed} and \textit{consistent} 
guarantee that the transitions is well-defined.
The observation table is \textit{closed} ensures that every row in the lower part also occurs in the uper part.
In other words, the row labeled by elements
of S are the candidates of states of the automaton.
\textit{Consistent} condition implies that both words lead to the same 
state in the automaton, as they cannot be distinguished by any $a \in \Sigma^*$.

The peseudocode \ref{alg:angluin_algorithm} presents Algluin's algorithm in pseudocode.
Essentialy, in the begin of learning process, the algorithm guarantees that the table are \textit{closed}
and \textit{consistent} by repeatly modifiding the columns and also the rows of the table.
After every extension of the table,
the algorithm fill the table by asking the membership queries for all table entries 
$u \in (R \cup R \cdot \Sigma) \cdot S$ for which no membership is yet present by asking the \textit{Teacher} the membership queries. If the \textit{Teacher} replies "yes",
then set $T(u) = 1$, otherwise  $T(u) = 0$.
Once this is the case, the observation table satifies the conditions, Angluin’s algorithm constructs a conjecture, which it submits
to an equivalence query. The learning terminates once the teacher replies “yes” on
an equivalence query.
However, if the Teacher returns a new counterexample $t \in \Sigma^*$ the algorithm modifies the table 
by adding t and its prefixes to S and repeats the process by going to line 1.
\begin{algorithm}
    \caption{Algluin's learning algorithm \cite{ANGLUIN198787}}\label{alg:angluin_algorithm}
    \textbf{Input: } A teacher for a regular language $L \subseteq \Sigma^*$
    
    Initialize the observation table (S, E, T)
    
    Ask membership queries for $\lambda$ and each $a \in \Sigma$
    
    Repeat:
        \begin{algorithmic}[1]
            \While{(S,E,T) is not closed or not consistent}
                \If{(S,E,T) is not consistent}
                    \State find $s_1$ and $s_2$ in S, and $e \in E$ such that                 
                    \State $row(s_1) = row(s_2)$ and $T(s_1 \cdot a \cdot e) \neq T(s_2 \cdot a \cdot e)$,            
                    \State add $a \cdot e$ to E,      
                    \State conducts membership queries.
                \EndIf

                \If{(S,E,T) is not closed}
                    \State find $s_1$ and $a \in \Sigma $ such that             
                    \State $row(s_1 \cdot a)$ is different from $row(s)$ for all $s \in S$,            
                    \State add $s_1 \cdot a$ to S,      
                    \State conducts membership queries.
                \EndIf
            \EndWhile
            \State Once (S, E, T) is closed and consistent, make $M = M(S,E,T)$
            \If{the Teacher replies with a counter-example t, then}
                    \State add t and all its prefixes to S         
                    \State conducts membership queries.
                \EndIf
        \end{algorithmic}
    Util the Teacher replies "yes"

    Terminate and return a conjecture $\mathcal{M}$
    \end{algorithm}

\subsection{NL*}
In general, a nondeterministic finite automata \textit{NFA} is often preferable to a 
deterministic finite automata \textit{DFA} due to potentially exponential 
differences in their sizes (REFERENCE FOR COMPARISON OF NFA AND DFA).
Therefore, learning algorithms for nondeterministic finite automata (NFA) are required.
In this section, we will introduce another active learning algorithm called the \textit{NL*} algorithm \cite*{Bollig2009AngluinStyleLO},
based on \textit{L*}. The \textit{NL*} concludes a residual finite-state automata (RFSA), 
a subclass of nondeterministic finite automata was introduced in the seminar work \cite*{10.1007/3-540-44693-1_13}.

Technically, it is possible to learn an RFSA instead of a DFA by modifying Angluin's algorithm $L^*$ observation table.
The proposed method involves selecting \textit{prime rows}\footnote{\label{notenl}\textit{prime row, RFSA-closed, RFSA-consitency} are defined in \cite{Bollig2009AngluinStyleLO}} as representations of the automaton's states, 
rather than utilizing \textit{all rows} of the table. 
The proceed of the \textit{$NL^*$} learning algorithm is mainly the same with \textit{L*}.
Similar to \textit{$L^*$}, it is also repeatedly checked the \textit{RFSA-closed}\footnoteref{notenl}
and \textit{RFSA-consitency}\footnoteref{notenl} properties, once the both properties are fullfill, 
it can contruct the conjecture and ask the equivalent query to the teacher.

\subsection{Kearns-Vazirani}
Another active learning algortihm is introduced in this thesis is \textit{Kearns and Vazirani's} \cite*{kearns1994introduction}.
Unlike \textit{Angluin's algorithm} it organizes its data in an ordered binary tree. 
It aims to minimize the number of membership queries by storing only one representative for 
each L-equivalence class in the tree.
The data are stored in two non-empty set $R, S \subseteq \Sigma$, where R consists
of \textit{representatives} that are used to represent the equivalence classes of L.
The set S includes \textit{separating words} that are used to verify that two 
different representatives indeed represent different equivalent classes.
More formally, \textit{Kearns-Vazirani's} algortihm keeps a separating word $v \in S$
for any two  representatives $u \ne u' \in R$ such that $uv \in L \Leftrightarrow uv \notin L$ is satisfied.

The organization of the binary tree is simple, while the inner nodes are labeled with 
the word of S, the leaf nodes are labled with words of R. The algorithm labels the root node's 
with $\epsilon \in S$.
The main property is that for each subtree, it places on the subtree's root $v \in S$
and all the $u \in R$ depending on whether $uv \in L$ or not. When $uv \notin L$ u is put in the left subtree.
Otherwise, $uv \in L$ u will be put in the right subtree. 
This procedure is recursively repeated at each subtree until all representatives
are put in their own leaf node.

The conjecture of \textit{Kearns-Vazirani's} algorithm is defined following: DFA $\mathcal{H} = (Q, \Sigma, q_o, \delta, F)$.
Where the set of states $Q = R$. The final states F consist of all representatives $u \in R$
that are located in the right subtree of the root node. 
Since $\epsilon$ is always an element of R, the initial state $q_0 = \epsilon$.
\subsection{Rivest-Schapire}
The last algorithm we will introduce in this thesis is \textit{Rivest and Schapire}.
The different of this algortihm to \textit{Angluin's algorithm} is,
that uses a \textit{reduced} version of Angluin's observation table
that stores exactly one representative per L-equivalence class.
The advantages are storing less data and asking less memberships queries.
(In fact, this method has originally been introduced by Schapire). 
\section{Libalf: the Automata Learning Framework}\label{section:libalf}
*\textit{Libalf} is a comprehensive, open-source program library
for learning finite automata. It was used for a large share of the
experiments conducted in this thesis, and many of the 
algorithms used or developed in later chapters habe between
integrated into the library.*
It supports both for active and passive algorithms but 
in this thesis we only consider the active algorithms.
The basis libraries that are required for libalf are libAmore and
libAmore++ that support for representing the automaton
such that NFA and DFA.

Internlly, \textit{libalf} represents words as list symbols
where each symbol is an integer data type. Thus, the maximal size
of an alphabet is $2^{32}$ or $2^{64}$ depending on the architecture of the 
target machine.
